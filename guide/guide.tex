\documentclass[10pt]{article}
\usepackage{palatino,url,parskip,alltt,multirow,graphicx}

\setlength{\hoffset}{-1.5in}
\setlength{\voffset}{-1.25in}
\setlength{\textheight}{10in}
\setlength{\textwidth}{7.5in}

\title{Civilization 4: BTS Guide To Peaceful Building}
\author{James Foucar}
\date{January 2017}
\begin{document}

\section*{Intro}

Being a good peaceful builder means being able to acheive a decisive demographic advantage
over your opponensts without going to war. This is extremely useful because wars are risky, have
diplomantic consequences, and the game mechanics greatly favor the defender. There are, of course,
risks to peaceful building as well. Sky-high demographics can scare other players and tempt them
to consider a dogpile on you.

The optimal scenario for the peaceful builder is to reach key military
techs well-ahead of other players in the renaissance/industrial era. The door is then open for conquest
or to double-down on peace building and going for runaway status. In the latter case, you try to stay
safe by scaring other players with your more advanced army and thus discouraging a pile.

This guide will explain in detail the key concepts to peaceful building:
* Choose an opener
* City specialization
* Leverage great people
* Optimizing cities, terrain improvements
* Key wonders
* Mid game
* Kicking into high-gear in later game

If you've understood the concepts in this guide, you should have no trouble beating the AI
on monarch difficulty.

Before reading this guide, you should have a basic knowledge of civ terms like 'specialist'
and 'happy cap'.

\section*{Choose an opener}

Much like chess, civ has a handful of useful opening (very early game) strategies. Once you
have a rough sense of your starting position, and considering your civ's strenths and weaknesses,
you should formulate your opening strategy, which will most likely be one of the following.

* Basic
* Rex (rapid expansion)
* Marble
* Stone
* Monk
* Trader
* Aggression opener
* Great people rush

One thing that's consistent in all openers is that forests should be aggressively chopped and
having lots of workers is absolutely essential. If you are blessed with very good land, you
might be able to pursue mutiple openers at the same time.

\subsection*{Basic}

This subsection contains things that should be considered regardless of your opener.

Turns 1-10:
* Beeline bronze working
* Scout the surrounding area, look for good city spots, neighbors and goodie huts
* ensure your capital has at least a warrior in it

Turns 10-40:
* Consider letting cap reach size 2 before building things that stall growth (worker, settler)
* Focus on getting a couple workers out. Goals are to chop forests and improve bonus tiles

Turns 40-60:
* Get an expansion or two. Top priority is metal, if no bronze, then top priority becomes horses
* Connecting cities gives a nice little boost to GNP, so be sure to invest in roads
* Each city should have at least one worker aiding it
* Continue chopping like crazy

Turns 60-80:
* You should have Writing
* You should have met at least one neighbor, try to get trade routes with them. Inter-civ trade is powerful in civ4, especially
  before cottages turn to towns.
* You should have at least one city, probably capital, starting to work cottages, preferably riverside cottages
* Figure out your capital's strength(s) and try to capitalize on them. If it has few hills, then it's probably
  going to focus on commerce. If it has a lot of hills and the food necessary to feed the miners, then you might
  consider going for some wonders.
* Continue to expand at a rate that keeps your science slider in the range of 20-60\%. If you drop below 20\%,
  you're on the verge of over-expanding and you need to stop. If you're above 70\% then you probably aren't
  expanding fast enough.
* New cities should prioritize granaries, monuments (if not creative), and workers
* Consider loaning workers to new cities to jump-start them

Turns 80-100:
* You should have several cities by this time.
* You're probably pushing up against health/happy caps in your bigger cities depending on what
  resources you have. You should select the tech path that most effectively alleviates the caps
  on your city growth.
* You should start paying close attention to neighbors' demographics to see if they are building up army
* start working more and more cottages

A thing you need to consider throughout is the opportunity cost of what you spend your hammers on. It is
hard to overstate how precious your early-game hammers are to your civ. Every hammer spent should be
towards something essential for your civ. Newer players tend to chase suboptimal wonders and/or make every city
improvement in every city. Your tech path should also be chosen very deliberately; if you aren't desperate
to have a tech, you should be researching something else.

A related item to consider is the 'peace dividend'. Don't put hammers into army unless you have to. Negotiate
non-aggression pacts with neighbors if possible so that both of you can invest in more productive things.

\subsection*{Rex}

Getting new cities quicky is something you'll want to do in almost any opener, but in this
opener you go all-in on expansion with a goal of 7+ cities by 0 AD. Your tech slider will take
quite a beating for a while, but once your cities start to get productive, things should snowball
into a decisive demographic lead going into mid-game.

Circumstances where Rex makes sense:
* Imperial, expansion, organized, or creative leader (in that order)
* Sumeria's early courthouse GODLIKE
* Having very ample land area available: lots of nearby land, few/faraway neighbors to claim it
* mediocre land
* presence of precious metals (allows you to afford more city maint)

Risks:
* Barbs
* Aggression from other players if you take their land
* Over-expansion

Benefits:
* Claims more land, duh! And land is generally the ultimate decider of a civ's power in later phases of the game
* Get EP leads on neighbors
* Per-city affects help you more than other players (trade, EP)
* Great synergy with religion

Often-times, if the map generator gives a generous allocation of land area to yourself,
that land will be fairly mediocre quality (large proportion of non-green land (plains, desert, tundra)).
This is the game telling you that it wants you to Rex. If you have both ample and high-quality land, you should
add an additional opener ontop of rex and go early for runaway status. I'll assume most players aren't
so lucky to find themselves in this position, so will tailor this guide to the former (mediocre land) case.
The idea here is that, if your land is mediocre, your demogs will not be competitive with other players unless
you have significantly more land and cities than they do.

The details of rexing are fairly simple. Your cities won't be building much other than workers, settlers, and
enough army to hold back barbs. Your first tech should be a beeline to broze working (you should almost
always do this regardless of strat).Your first city should build: warrior, worker, worker, worker (3rd worker if heavily
forested), warrior (to escort setter), settler. You'll repeat the escort + settler build several times. Your first city
should grab metal if you don't have any in your cap so you can start making metal-units as your escorts. You should settle
in the direction of your nearest neighbor first if there's a risk of losing that land race. Otherwise, settle the best spots
near your capital first in order to minimize the maintenace hit. Your workers should most be chopping like crazy and building your
road network, only grow your cities big enough to work bonus tiles. If possible, try to put chops into workers/settlers, and on
turns where there's no incoming chop bonus, make your escorts. This micro-management can allow your cities to grow at least a
little bit even while making huge numbers of settlers and workers. New cities will need a granary and monument (if not creative).

The main challenge with rexing is not completely crashing your economy with city maintenance costs. One thing to note is
your cities should remain on the smaller side due to all the settler and worker building, so switching to monarchy is not as high of a priority.
Instead, your main tech thrust after you have the basics is courthouses. They are absolutely essential. The other key
thing is to at least get internal trade routes set up very quickly after a city is founded. Even better is getting OB with
your neighbors for the foreign trade routes. Keep in mind that trade routes are a per-city boost and hence will benefit a
rexxing civ more.

If you've rexxed to the point that, even with courthouses and foreign trade routes, your slider is dipping below 30\%, you
need to stop rexxing. You do not want to fall into the trap of having no tech rate and nothing useful to build in your cities
except more units. If you have to, take all your citizens off of production tiles and work commerce tiles or merchant specialists.
Your primary tech target after courthouses is currency for the beautiful +1 trade routes per city. Again, as a rex player, this boost is even
better for you than other players. I've found that once I have currency and a few markets down in my best cities, I can Rex
to my heart's content thereafter.

With lots of cities and lots of courthouses, you should be an EP power. If you focus all this EP on one of your neighbors, you should
find yourself in a position to steal a couple techs by mid-game. Tech stealing can help you to catch up since you're almost certainly
a little behind the top players in tech due to running a lower tech slider.

As you transition into mid-game, you should note that the rex strat has a very nice syngery with religion. This is because
the shrine and the religious wonders (AP, miranet, and sankore) all confer bonuses that scale per-city. You should probably consider
trying to get one of these, especially if you have access to stone. Note you don't have to have founded a religion to
be a good monk player (you probably didn't due to the low tech slider early on).

A perfect example of leveraging a mediocre-land Rex for an easy win is Jim's NDSMiniI win.

SHOW SCREEN OF MINI1 layout

\subsection*{Marble}

I generally discourage players from paying full retail value for a wonder (building a wonder without
any production bonuses) unless you've formulated a strategy in which a specific wonder is essential.
Almost all the early wonders in the game benefit from either marble or stone. Stone and marble provide
an incredible 100\% production boost to the wonders that they boost, making the wonder much more
attractive. This is important enough that the presence of marble or stone in or near your capital
will have an impact on your opening strategy.

The classic pattern if you have access to marble is to make an early push for masonry (allows you to mine marble)
and priesthood (unlocks Oracle) in order to get the Oracle. Oracle allows you to instantly research a tech, the most
likely candidate being Metal Casting (MC) which an insanely expensive tech for how early it is in the tree and it's also
extremely useful. The Forge city improvement that MC unlocks is excellent for boosting production and it also
gives you very early access to the Colossus wonder which virtually guarantees that you'll be the first to build it if
you choose to do so. Colossus is a good, not great, wonder if you have a large number of coastal city spots in your
land deal.

The next wonders you should push hard for are the world wonders and national wonders unlocked by the Aesthetics and
Literature tech path. A strong move for a player with marble is to make the Great Library (GL) and then put the national
epic in the same city (both are boosted by marble). This will guarantee you several great scientists in the next
50 turns or so, even more if you are philosophical. Parthenon (also marble-boosted) is nice to have, but not nearly as game changing as
GL.

If you can pull off all of the above while also doing a reasonable job of growing your civ, you should be
in excellent shape headed into mid-game. Your deluge of scientists should put you at or near the head of the pack
in the tech race. Consider going hard for education and then liberalism. Another possibiliy is going for Masoleum (calendar)
if you plan on going heavy into specialists (beyond the two freebies from GL).

The approach has nice synergies with the philosophical and industrial traits (cheaper wonders AND forges).

\subsection*{Stone}

Same story as marble except you have a different set of wonders which are boosted. The obvious game changer
here are the Pyramids (mids) because it unlocks the Representation civic which gives all your specialists +3
beakers and provides +3 happy to your 5 biggest cities which should instantly alleviate all concerns over
your early-game happiness cap.

Capitalizing on stone is a bit more challenging than using marble because you need to run specialists to
take advantage of being in Representation. If you're able to secure the mids, you should hurry to set up
at least one specialist-focussed city (a city that generates food and then uses the population fed by that
food as specialist, these cities generally work farms instead of cottages).

The library improvement is useful here as it unlocks two scientist specialist slots. With representation,
you can turn these two slots into an amazing (for early game) 12 base beakers. This alone will give you
the beaker-horsepower to get an early tech lead. Since stone boosts the excellent mid-game religious wonders
(miranet, sankore), you should consider going the 'monk' route here see below.

This approach has nice synergies with philo (since you'll be running specialists), industrial (you'll be
going for at least 3-4 wonders), and spiritual (free early switch to representation).

\subsection*{Monk}

Monk players leverage religion to become powerful. This means founding a religion is a top priority
after you get bronze working. Ideally you'll get multiple religions, making it more likely that your
neighbors will want to get and spread one of your religions. Having an early religion is nice in that
it allows you to skip monuments since the religion alone gives you 1 cuture/turn. Use the hammers you would
have spent on monuments to make missionaries instead. Offer missionaries to your neighbors; they will want
your religion if they don't already have one. You should leverage the preist slot opened up by temples
to generate a great prophet in order to make a shrine of your best religion. Also, note that two monasteries
gives the equivalent tech boost as a library. If you have a good shrine and mutiple monasteries in your
best commerce cities, then you should have a very strong economy, which means you have a chance to get
to key techs before other players if you beeline them. The obvious tech to beeline once you've established
the basics is Theology. This will give you yet another religion (christianity) but more importantly, it will
allow you to construct the game-changing Apostolic Palace (AP). Beyond the very complex diplo options unlocked
by the AP, the AP gives +2 base hammers to each building matching the religion of the AP. Since you should have
a temple and monastery in all your cities, this is like getting a free plains-hill mine in all your cities.
Miranet and Sankore should be pursued after the AP.

\subsection*{Trader}

As mentioned 

\end{document}
