\documentclass[10pt]{article}
\usepackage{palatino,url,parskip,alltt,multirow,graphicx}

\setlength{\hoffset}{-1.5in}
\setlength{\voffset}{-1.25in}
\setlength{\textheight}{10in}
\setlength{\textwidth}{7.5in}

\title{Civilization 4: BTS Guide To Peaceful Building}
\author{James Foucar}
\date{January 2017}
\begin{document}

\section*{Intro}

Being a good peaceful builder means being able to acheive a decisive demographic advantage
over your opponensts without going to war. This is extremely useful because wars are risky, have
diplomatic consequences, and the game mechanics greatly favor the defender. There are, of course,
risks to peaceful building as well. Sky-high demographics can scare other players and tempt them
to consider a dogpile on you.

The optimal scenario for the peaceful builder is to reach key military
techs well-ahead of other players in the renaissance/industrial era. The door is then open for conquest
or to double-down on peace building and going for runaway status. In the latter case, you try to stay
safe by scaring other players with your more advanced army and thus discouraging a pile.

This guide will explain in detail the key concepts to peaceful building:
* Choose an opener
* City specialization
* Use the right civics
* Kicking into high-gear in mid-to-late game

If you've understood the concepts in this guide, you should have no trouble beating the AI
on monarch difficulty.

Before reading this guide, you should have a basic knowledge of civ terms like 'specialist'
and 'happy cap'.

\section*{Choose an opener}

Much like chess, civ has a handful of useful opening (very early game) strategies. Once you
have a rough sense of your starting position, and considering your civ's strenths and weaknesses,
you should formulate your opening strategy, which will most likely be one of the following.

* Basic
* Rex (rapid expansion)
* Marble
* Stone
* Monk
* Trader
* Great people rush
* Aggression opener

One thing that's consistent in all openers is that forests should be aggressively chopped and
having lots of workers is absolutely essential. If you are blessed with very good land, you
might be able to pursue mutiple openers at the same time.

\subsection*{Basic}

This subsection contains things that should be considered regardless of your opener.

Turns 1-10:
* Beeline bronze working
* Scout the surrounding area, look for good city spots, neighbors and goodie huts
* ensure your capital has at least a warrior in it

Turns 10-40:
* Consider letting cap reach size 2 before building things that stall growth (worker, settler)
* Focus on getting a couple workers out. Goals are to chop forests and improve bonus tiles

Turns 40-60:
* Get an expansion or two. Top priority is metal, if no bronze, then top priority becomes horses
* Connecting cities gives a nice little boost to GNP, so be sure to invest in roads
* Each city should have at least one worker aiding it
* Continue chopping like crazy

Turns 60-80:
* You should have Writing
* You should have met at least one neighbor, try to get trade routes with them. Inter-civ trade is powerful in civ4, especially
  before cottages turn to towns.
* You should have at least one city, probably capital, starting to work cottages, preferably riverside cottages
* Figure out your capital's strength(s) and try to capitalize on them. If it has few hills, then it's probably
  going to focus on commerce. If it has a lot of hills and the food necessary to feed the miners, then you might
  consider going for some wonders.
* Continue to expand at a rate that keeps your science slider in the range of 20-60\%. If you drop below 20\%,
  you're on the verge of over-expanding and you need to stop. If you're above 70\% then you probably aren't
  expanding fast enough.
* New cities should prioritize granaries, monuments (if not creative), and workers. Later on, courthouses and forges are mandatory.
* Consider loaning workers to new cities to jump-start them

Turns 80-100:
* You should have several cities by this time.
* You're probably pushing up against health/happy caps in your bigger cities depending on what
  resources you have. You should select the tech path that most effectively alleviates the caps
  on your city growth.
* You should start paying close attention to neighbors' demographics to see if they are building up army
* start working more and more cottages

A thing you need to consider throughout is the opportunity cost of what you spend your hammers on. It is
hard to overstate how precious your early-game hammers are to your civ. Every hammer spent should be
towards something essential for your civ. Newer players tend to chase suboptimal wonders and/or make every city
improvement in every city. Your tech path should also be chosen very deliberately; if you aren't desperate
to have a tech, you should be researching something else.

A related item to consider is the 'peace dividend'. Don't put hammers into army unless you have to. Negotiate
non-aggression pacts with neighbors if possible so that both of you can invest in more productive things.

\subsection*{Rex}

Getting new cities quicky is something you'll want to do in almost any opener, but in this
opener you go all-in on expansion with a goal of 7+ cities by 0 AD. Your tech slider will take
quite a beating for a while, but once your cities start to get productive, things should snowball
into a decisive demographic (food and hammers especially, GNP may still lag) lead going into mid-game.

Circumstances where Rex makes sense:
* Imperial, expansive, organized, or creative leader (in that order)
* Sumeria's early courthouse GODLIKE
* Having very ample land area available: lots of nearby land, few/faraway neighbors to claim it
* mediocre land
* presence of precious metals (allows you to afford more city maint)

Risks:
* Barbs
* Aggression from other players if you take their land
* Over-expansion

Benefits:
* Claims more land, duh! And land is generally the ultimate decider of a civ's power in later phases of the game
* Get EP leads on neighbors
* Per-city affects help you more than other players (trade, EP)
* Great synergy with religion

Often-times, if the map generator gives a generous allocation of land
area to yourself, that land will be fairly mediocre quality (large
proportion of non-green land (plains, desert, tundra)).  This is the
game telling you that it wants you to Rex. If you have both ample and
high-quality land, you should add an additional opener ontop of rex
and go early for runaway status. I'll assume most players aren't so
lucky to find themselves in this position, so will tailor this guide
to the former (mediocre land) case.  The idea here is that, if your
land is mediocre, your demogs will not be competitive with other
players unless you have significantly more land and cities than they
do.

The details of rexing are fairly simple. Your cities won't be building
much other than workers, settlers, and enough army to hold back
barbs. Your first tech should be a beeline to broze working (you
should almost always do this regardless of strat).Your first city
should build: warrior, worker, worker, worker (3rd worker if heavily
forested), warrior (to escort setter), settler. You'll repeat the
escort + settler build several times. Your first city should grab
metal if you don't have any in your cap so you can start making
metal-units as your escorts. You should settle in the direction of
your nearest neighbor first if there's a risk of losing that land
race. Otherwise, settle the best spots near your capital first in
order to minimize the maintenace hit. Your workers should most be
chopping like crazy and building your road network, only grow your
cities big enough to work bonus tiles. If possible, try to put chops
into workers/settlers, and on turns where there's no incoming chop
bonus, make your escorts. This micro-management can allow your cities
to grow at least a little bit even while making huge numbers of
settlers and workers. New cities will need a granary and monument (if
not creative).

The main challenge with rexing is not completely crashing your economy
with city maintenance costs. One thing to note is your cities should
remain on the smaller side due to all the settler and worker building,
so switching to monarchy is not as high of a priority.  Instead, your
main tech thrust after you have the basics is courthouses. They are
absolutely essential. The other key thing is to at least get internal
trade routes set up very quickly after a city is founded. Even better
is getting OB with your neighbors for the foreign trade routes. Keep
in mind that trade routes are a per-city boost and hence will benefit
a rexxing civ more.

If you've rexxed to the point that, even with courthouses and foreign
trade routes, your slider is dipping below 30\%, you need to stop
rexxing. You do not want to fall into the trap of having no tech rate
and nothing useful to build in your cities except more units. If you
have to, take all your citizens off of production tiles and work
commerce tiles or merchant specialists.  Your primary tech target
after courthouses is currency for the beautiful +1 trade routes per
city. Again, as a rex player, this boost is even better for you than
other players. I've found that once I have currency and a few markets
down in my best cities, I can Rex to my heart's content thereafter.

With lots of cities and lots of courthouses, you should be an EP
power. If you focus all this EP on one of your neighbors, you should
find yourself in a position to steal a couple techs by mid-game. Tech
stealing can help you to catch up since you're almost certainly a
little behind the top players in tech due to running a lower tech
slider.

As you transition into mid-game, you should note that the rex strat
has a very nice syngery with religion. This is because the shrine and
the religious wonders (AP, miranet, and sankore) all confer bonuses
that scale per-city. You should probably consider trying to get one of
these, especially if you have access to stone. Note you don't have to
have founded a religion to be a good monk player (you probably didn't
due to the low tech slider early on).

A perfect example of leveraging a mediocre-land Rex for an easy win is Jim's NDSMiniI win.

SHOW SCREEN OF MINI1 layout

\subsection*{Marble}

I generally discourage players from paying full retail value for a
wonder (building a wonder without any production bonuses) unless
you've formulated a strategy in which a specific wonder is essential.
Almost all the early wonders in the game benefit from either marble or
stone. Stone and marble provide an incredible 100\% production boost
to the wonders that they boost, making the wonder much more
attractive. This is important enough that the presence of marble or
stone in or near your capital will have an impact on your opening
strategy.

The classic pattern if you have access to marble is to make an early
push for masonry (allows you to mine marble) and priesthood (unlocks
Oracle) in order to get the Oracle. Oracle allows you to instantly
research a tech, the most likely candidate being Metal Casting (MC)
which an insanely expensive tech for how early it is in the tree and
it's also extremely useful. The Forge city improvement that MC unlocks
is excellent for boosting production and it also gives you very early
access to the Colossus wonder which virtually guarantees that you'll
be the first to build it if you choose to do so. Colossus is a good,
not great, wonder if you have a large number of coastal city spots in
your land deal.

The next wonders you should push hard for are the world wonders and
national wonders unlocked by the Aesthetics and Literature tech
path. A strong move for a player with marble is to make the Great
Library (GL) and then put the national epic in the same city (both are
boosted by marble). This will guarantee you several great scientists
in the next 50 turns or so, even more if you are
philosophical. Parthenon (also marble-boosted) is nice to have, but
not nearly as game changing as GL.

If you can pull off all of the above while also doing a reasonable job
of growing your civ, you should be in excellent shape headed into
mid-game. Your deluge of scientists should put you at or near the head
of the pack in the tech race. Consider going hard for education and
then liberalism. Another possibiliy is going for Masoleum (calendar)
if you plan on going heavy into specialists (beyond the two freebies
from GL).

The approach has nice synergies with the philosophical and industrial traits (cheaper wonders AND forges).

A good example is Jim's game 7 opening play.

\subsection*{Stone}

Same story as marble except you have a different set of wonders which are boosted. The obvious game changer
here are the Pyramids (mids) because it unlocks the Representation civic which gives all your specialists +3
beakers and provides +3 happy to your 5 biggest cities which should instantly alleviate all concerns over
your early-game happiness cap.

Capitalizing on stone is a bit more challenging than using marble because you need to run specialists to
take advantage of being in Representation. If you're able to secure the mids, you should hurry to set up
at least one specialist-focussed city (a city that generates food and then uses the population fed by that
food as specialist, these cities generally work farms instead of cottages).

The library improvement is useful here as it unlocks two scientist specialist slots. With representation,
you can turn these two slots into an amazing (for early game) 12 base beakers. This alone will give you
the beaker-horsepower to get an early tech lead. Since stone boosts the excellent mid-game religious wonders
(miranet, sankore), you should consider going the 'monk' route here see below.

This approach has nice synergies with philo (since you'll be running specialists), industrial (you'll be
going for at least 3-4 wonders), and spiritual (free early switch to representation).

\subsection*{Monk}

Monk players leverage religion to become powerful. This means founding
a religion is a top priority after you get bronze working. Ideally
you'll get multiple religions, making it more likely that your
neighbors will want to get and spread one of your religions. Having an
early religion is nice in that it allows you to skip monuments since
the religion alone gives you 1 cuture/turn. Use the hammers you would
have spent on monuments to make missionaries instead. Offer
missionaries to your neighbors; they will want your religion if they
don't already have one.

You should leverage the preist slot opened up
by temples to generate a great prophet in order to make a shrine of
your best religion. Also, note that two monasteries gives a nearly
equivalent tech boost as a library. If you have a good shrine and
mutiple monasteries in your best commerce cities, then you should have
a very strong economy, which means you have a chance to get to key
techs before other players if you beeline them. The obvious tech to
beeline once you've established the basics is Theology. This will give
you yet another religion (christianity) but more importantly, it will
allow you to construct the game-changing Apostolic Palace (AP). Beyond
the very complex diplo options unlocked by the AP, the AP gives +2
base hammers to each building matching the religion of the AP. Since
you should have a temple and monastery in all your cities, this is
like getting a free plains-hill mine in all your cities.  Miranet and
Sankore should be pursued after the AP.

This approach is best used with a civ that starts with the mysticism tech as this will give you easier access
to the game's early religions. Spiritual trait is nice for the free religion switching and cheap temples.
Philo is nice for quicker generation of great prophets. Industrial can help since this strat goes for at least
a couple wonders.

\subsection*{Trader}

Trade routes can singlehandedly carry your economy in the early game given the right starting conditions.
The optimal situation is starting on an island or coast-heavy starting spot on a continent in which there are coastal trade
paths to other neighboring civs. The goal is to establish a greater number of, and higher value, trade routes
compared to other civs.

The game-changing wonder here is the Great Lighthouse (GLH) and the
tech that unlocks this wonder (sailing) should be beelined immediately after bronze working. GLH provides
+2 trade routes in every coastal city, so settling coastal cities should be a top priority after GLH has
been built. You should also focus on exploration to meet neighbors and establish open borders (OB) with them.
The Temple of Artemis (x2 trade route value in city) is nice to have but not necessary for this opener.
If you have marble or remaining forests, go for Artemis. After which, getting writing in order to unlock Open
Borders is a top priority if you've met some neighbors. After that, currency (+1 trade routes in all cities) and
then compass (harbors +50\% trade route yield) are priorities.

Remember that your civ can have at most 1 trade route to any other players' individual cities in your civ and you'll want the
majority of your trade routes to go to other civs (foreign trade is higher-value). This means that, once you have GLH
and currency (4 trade routes in coastal cities), you need to have open borders and trade connections to roughly 4
other civs (assuming all civs are roughly equal size). This will require you to aggresively explore the land and ocean
and also have good diplomatic skill. Other players should know that you are benefiting from OB more than they are
and may ask for concessions for OB.

If you are able to pull this strat off, your GNP will be very dominant. Other players may notice this and be tempted to
close borders, so you may need to bully or bribe them to maintain OB with you. War is very costly for the Trader civ since
it reduces trade possibilities, so war should be avoided until the Renaissance era if possible. Your high GNP should translate
into a decisive tech lead by this point.

The use of this strat is entirely driven by your land deal and is mostly independent of your leader and civ. Industrial
can help you build the GLH faster and organized can help you make lighthouses faster (a lighthouse is required in the
city that builds GLH).

A perfect example of this strategy in use is Jim's game 6 opening play.

\subsection*{Great Person Rush}

Unlike most openers, this one is mostly driven by your choice of civ. The only land-deal requirement is a capital
or nearby expansion spot that's food-heavy. Most land-deals should be able to meet this requirement. The key thing
is to make the right decisions when picking your leader and civ. Obviously, your leader should be Philosophical. The
other key thing is that you pick a civ with an early-game uber-building that features extra specialist slots. There are
many civs that meet this requirement with the best probably being Egypt.

Your opening play is pretty straightforward: once the basics are taken care of, rush your uber-building, work your food
heavy tiles, and put the surplus population into your specialist slots. You'll be generating a huge number more great people
than other civs and most of these great people should be settled in your best city.

If you can pull off the above and also nab the mids, you'll be in
fantastic shape and possible already a runaway. If you're lucky enough
to have stone nearby, you should definitely take a shot at
mids. Another decent wonder for this strat is the masoleum of massolos
(+50\% golden age length and golden ages give +100\% great person
generation). It's a fairly mediocre wonder in general, but since
you'll be generating a lot of great people, you'll have the potential
to pop at least a couple golden ages.

Another high priority is the National Epic (national wonder) which will further buff your great person generation. Later
in the game, consider the Pacifism civic if your situation looks peaceful. If you missed the mids earlier, then
constituion (for representation civic) is a high priority.

Synergies: charismatic for the +2 happy. You'll be wanting to grow your great person city large in order to run lots
of specialists. Charismatic can help you avoid running into the happy cap.

A great example of this opener is Pat's game 8 win.

\subsection*{Aggression}

The civ gods can be cruel and sometimes you get dealt a bad starting land. The land can be poor quality (lacking
green tiles) or in quantity (small island or close neighbors). In this case, there's not much point in trying
to peacefully development because you won't be able to keep up with civs that got better land. Fortunately, in most cases,
poor land deals usually have access to war resources (bronze, iron, and horses). You should build a couple expansions
to increase you war output, link strategic resources, and then beeline military techs.

This strat synergizes well with civs that have an early-game uber unit and it also goes well with the Aggressive trait.

Since this is a peaceful development guide, there's not much more to say here.

A good example of this strat is Jim's early game rush of Aaron in game 5.

\section*{City Specialization}

A big part of peaceful development is building the right city improvements (buildings)
in the right cities at the right time. You'll also want at least a few different
kinds of cities in your empire in order to maximize your effectiveness. It's OK
to have most of your cities be standard all-around solid contributors (solid commerce,
hammers, and food) (hereafter called standard cities) but you don't want all of
your cities to be this way.

City types:
* Standard
* Commerce
* Production
* Super science
* Super gold
* Great person farm
* Military
* Resource colony

\subsection*{Standard}

Most of your cities will be standard cities or at least start out that
way if the improvements/national-wonders that it needs to be some other type
are not available yet. In most land deals, your capital should have exceptionally
good land, so it should not become a standard city. Cities with decent, but not
great, potential typically become standard cities. A city's potential is almost
always closely related to how much \"easy\" food it has access to (flood plains
and food bonus resources (corn, pig, etc). A standard city will have access to
1-2 strong food nodes (wet corn, grassland pig, fish) or 2-3 weaker food nodes
(wheat, sheep on a hill, cow) and will have at least a couple green tiles.

Once your easy food is developed, the city will have enough surplus to work 2-3
mines and 2-3 cottages in the early/mid game. So, 2-3 citizens producing food,
2-3 citizens producing hammers, and 3-4 citizens working cottages depending on
how many grassland tiles are in your city.

Standard cities are nice but not great. They require nearly all of city improvements
(health/happy building to allow growth, barracks in case you go to war, forges/factories to boost production,
and libraries/markets etc to capitalize on the cottages) without providing
any real game-changing benefits. In other words, you invest a lot and only get
a modest amount back. Still, the investment is positive so it's well worth the effort.
In the lategame, with factories and corporations, pretty much any city can become
very productive.

\subsection*{Commerce}

Commerce cities are not that different than standard cities with the key difference
being that they can support a great number of towns. A classic commerce city will
have enough easy food to work 2-4 hills/workshops and a LOT of flat grassland (green).
The wonderful thing about a grassland cottage is that it feeds itself (2 food) and
yields excellent commerce once the cottage develops. As long as the city grows, if it
has a fresh grassland cottage to work, then it will continue to grow at the same pace.
It's not unheard of to get a city working 10+ cottages if it has a lot of flat grassland
tiles. This kind of city will yield a massive amount of commerce in the mid to late game
and can help you gain a tech lead.

Commerce cities require most city improvements just like standard cities. If the city
has extremely poor production (1-2 hills or workshops) you can consider skipping war
buildings like barracks. The city will never contribute meaningfully to your army and
will spend all its hammers on buildings that boost commerce (markets, libraries, etc).
Commerce cities on the coast may also derive significant commerce from trade routes.
If so, buildings like harbors and customs house are high priorities.

Once your great person (GP) farm is up, you should get a great scientist and drop an
academy in your commerce cities.

In the later game, Universal Suffrage can turn hammer-starved commerce cities into
solid production cities (+1 hammer for town). This is well worth a beeline if you're
heavy on commerce cities. Commerce cities are also best on/near rivers, so the levee
improvement can significantly boost hammers. These things need to be considered because
commerce cities tend to be hammer-poor and hammers are very important.

\subsection*{Production}

Production cities focus on generating hammers. In the early-to-mid game, this means
working lots of mines. The best case is a city with 2-3 food resources and a large
number of grassland hills. Grassland hills are wonderful because they generate 1 food
and therefore only cause a net deficit of 1 food. A city with 2 wet corn can support
a whopping eight grassland hill mines. In reality, you probably won't be that lucky,
but hopefully you'll at least have mix of grassland hills and plains hills. The plains
hills have an additional hammer but generate no food so they will quickly deplete
your food surplus. If you have a large number of plains hills, windmills should be
considered because it will be impossible to feed that many plains-hill miners.

The concept here is pretty simple, use food resources to feed miners. Throw in some
cottages too if the land allows for it to make a production/standard hybrid city.
The more important question is what to do with the hammers? The options are, make
wonders, make army, or make useful civ-boosting units like workers, settlers, and missionaries.
If the city generates decent commerce, you should be sure to
build most city improvements. If not (city lacks usable flat tiles due to desert,
tundra, ice, ocean), many city improvements can be skipped.

If you tech rate is decent, a good production city should allow you to take a shot
a some wonders and have a high chance of success. If your are able o build several
wonders in one city, it will become a decent GP generator, but it will NEVER be your
GP farm.

\subsection*{Super science}

A super science (SS) city is where you will be building your Oxford
University. An effective SS city will use commerce and specialists to
generate huge numbers of beakers. A classic mid-game example of a
super science city is a commerce-heavy capital boosted by Bureaocracy
a couple of science specialists, some settled great scientists, and an
academy. Even better if the player is running representation. Such a
city may be generating as many beakers as the rest of your civ
combined, hence its name. Due to the awesome power of Bureaocracy, your
SS city will probably be your capital unless it has poor commerce
potential.

Once your GP farm is up, generating loads of great scientists and settling
them in your SS city is a solid play (after SS city has academy of course),
especially if you're running representation.

It's probably worthwhile to make gold-boosting city improvemnts (markets, banks)
in your SS city since the city probably has a ton of commerce.

\subsection*{Super gold}

A super gold (SG) city is where you'll be building your Wall Street. Since
most players run a high beaker slider, it's not as important that your SG
city has great commerce, although that's nice of course. What's much more
important is that your SG city one or more of the following: a large number
of gold-generating specialists and super-specialists, a shrine of a common
religion, or a corporation(s) headquarters. These three things generate
gold regardless of your tech slider, so you SG remains a huge gold producer
independent of the sliders and therefore allowing you to push to extreme-high
tech slider (80+\%). Settling great merchants and great prophets in your SG
city is a great play.

If your SG has low commerce, it may not be worthwhile to make the beaker-booster
improvements, or at least make them low priority. Note that your SG city
is usually NOT your capital and it is almost NEVER the same city as your SS city.

\subsection*{Great person farm}

Great people (GP) are extremely powerful in civ4 and in order to get large numbers of
GPs and to get the types of GPs you need, you'll need a GP farm. Your GP farm is where
you'll be putting your National Epic (NE). If you've been paying close attention for the
last three subsections, you'll notice that I've put a heavy emphasis on three national
wonders: oxford, wall street, and national epic. That is not an accident as these three
national wonders are game changer and are available to all civs. It is important that you
have a plan for where these three national wonders will go well before you actually build
them.

The perfect GP farm city will have loads of easy food and lots of flat grassland tiles for
farms, allowing it to generate a massive food surplus. It will also have 3-4 plains-hill tiles
for production with production is needed. The city should work it's mines until the national
epic is complete AND the specialist slots it needs are available. With the awesome Caste System
civic, you'll already have unlimited artist, scientist, and merchant slots. Once the slots are
available and NE is complete, all citizens should be taken off of production tiles like mines
and be used as specialists instead. The city should be generating massive amounts of specialist
points and you'll have full control over the type of GPs you generate. You should avoid polluting
your GP point pool by building other wonders in this city. Having precise control over the GP
points being generated is almost as valuable as the points themselves. Ask any player who missed
their great merchant at 75\% odds and therefore missed Sushi co.

You should only turn hammers back on in emergencies or to rush important city improvements.

\subsection*{Military}

A city with decent hammer potential but nothing else can still become a very important
contributor to your civ. I'm thinking of a polar city with a fish and three tundra hills, one with
iron. It will never be productive enough to make wonders and has almost zero commerce potential.
Turn this city into a military city and you'll still love it. The great thing about military cities is
that they need very few city improvements to do their job. They need a granary, courthouse, forge, barracks,
stable in the early-to-mid game, that's it. All the rest of the hammers can be used to pump units non-stop.
This ensures that your army stays competitve and frees up your other cities to peacefully develop. You can
put your herioc epic and later West Point in this city. Settle surplus generals here if you're fighting a lot.

\subsection*{Resource colony}

Some cities have zero potential but still need to be placed in order to pick up a critical resource, like
a deep polar iron or oil city. You can slowly build a courthouse then have the city build wealth for the
rest of the game to generate a few extra gold a turn.

\section*{Civics}

A common mistake for newer players is to run out-of-date civics. Running the right civics
for your civ is extremely important and the techs that unlock the best civics for your civ
are well-worth beelining. Let's look at some of the key early-game civics.

\subsection*{Slavery}

The first civic you'll unlock is slavery, one of the more complex civics in the game. Even very
new players should understand that slavery is a great wartime civic because it gives you the
ability to \"panic whip\" a large number of units in a very short time. Basically, when your
civ's safety is on the line, you switch all cities to unit production and then whip out a unit
in every city every couple turns. This should only be done in a panic situation (surprise invasion
by a neighbor) because it will take a huge toll on your population.

More advanced players can use slavery to boost production, especially in production-poor cities.
Once workshops enter the scene and get boosted by various techs, almost any city with decent
food access should be able to achieve decent production, but before that, some cities may have
decent potential but almost no production due to a lack of hills. I'm thinking of a coastal city
with a couple fish and several grassland flat tiles. This city has potential due to the fish, but
it will only have 1 base hammer generated per turn until workshops are available. You cannot have
a city that builds nothing for the first 100 turns, so you need to get production from somewhere and
that somewhere is slavery. The correct play is to work the fish, allow the city to grow, then whip
out the current city improvement, wait 10-15 turns and repeat. Small cities with good food surplus
and a granary will very quickly regrow the population lost from whipping.

\subsection*{Monarchy}

Hereditary Rule (monarchy) is probably the most important civic in the very early game. Most players,
especially players are not charismatic, lacking in precious metals, and missing the early game religions,
will find that the growth of their cities is very quickly stymied by happiness problems. The monarchy civic
is inexpensive (low upkeep) and very useful because it provides a path for players to completely
eliminate happiness problems in their civ. All you have to do is station an additional military unit in a city
to get +1 happiness in that city. You don't want to get too carried away with this as it is expensive in
hammers and unit-maintenance to have a large army, but it can be very helpful to allow your best cities
to keep growing. The monarchy tech is pretty expensive but well worth a high prioritization or beeline
in the early game if you have a very low happiness cap.

The only cases where Monarchy is not a high priority is if you're loaded with happiness resources. It's not
unheard of, especially when trading with other players, to have access to 2 precious metals and a couple
calendar happy resources (sugar, dye, etc). With a forge, the 2 precious metals become +4 happy, making
monarcy a much lower priority. The other case is if I've built the pyramids and am therefore, of course,
going to run representation instead of monarchy.

\subsection*{Organized Religion}

Organized Religion (OR) is an interesting civic. It provides the equivalent of a forge in all cities that
have your state religion when making a building, so it may be attractive to a monk player in the early game or other players
in the mid game. The problem with OR is its high upkeep. If you civ is organized, that's not a problem, but if not, and
especially if you are trying to rex, then the cost of OR can be a problem. If you can afford to run OR in
the early/mid game, you should.

\subsection*{Caste System}

Caste System (CS) is a fanatastic civic and I usually run it for most of the game. It provides two fantastic benefits,
+1 hammer for workshops and unlimited specialist slots in all cities for artists, scientists, and merchants. It comes a
little later that the civics mentioned above, but not much because code-of-laws (tech to unlock CS) is such a high
priority for most players. With CS and guilds, workshops become as good as mines. CS is also essential for setting up
your GP farm city.

\subsection*{Bureaucracy}

Bureaucracy (bureau) is the last of what I would call the early game civics and it's very very good, especially
if you've rolled a strong capital. Bureau will catapult your demogs upwards more than any other civic, providing
+50\% to BASE commerce and hammers in your cap. The same tech (civil service) partly unlocks maceman too, so this
tech is usually very high priority for most players. Bureau will transform a commerce-oriented capital into a super-science
beakerfest and will transform a production-oriented capital into a wonder-churning production machine.

\section*{Mid to Late Game}

As stated in the introduction, the goal of the peaceful builder is to have a decisive tech lead by the late rennaissance / early industrial
era (rifling, steel, etc). This is when the boosts compared to the previous era start to become enormous and the
game begins to snowball. If you've reached this point with a significant lead, you can use the superior units of this era
to easily conquer neighbors and swing into domination mode. The demographic boosts in this era are also massive, so you can
also double-down on peace to start to build an enormous demographic lead over other civs. Since this is a peace guide, we will
explore the latter.

This era has a number of first-to tech bonuses that are very powerful, the best being the liberalism (lib) tech since it allows you
to choose a tech for free upon completion. The liberalism beeline is a very classic play for the player who's in the tech
lead. You can use it to unlock astronomy if dominating trade and the seas is a priority. If you have marble, nationalism is
a reasonable choice to unlock the Taaj Mahal wonder for a free golden age. There are also a number of strong military techs in
this region of the tech tree, so the correct choice will depend on the situation.

Economics is another great tech to prioritize as it provides a great merchant if you are first-to and it unlocks the excellent
Free Market civic. It's not as good as liberalism but can be a good consolation prize if you miss lib.

You should already be thinking about the corporations you are going to found. Mining is the best in the game in almost all
situations, but the great engi can be hard to get. Sushi and Cereal are excellent if you have lots of fish or grains respectively and only
require a great merchant which should be easy to get (you have a GP farm, right?). When you have a tech lead, you should be able
to get the best corporations and this is a very key part of the late game snowball.

If you have a ton of farms, then biology into physics for the +1 food to farms and free great scientist is probably
the best option, otherwise go for the game-changing factories. Factories are good on their own (+25\% hammers) but become
godly with a power source (an additional +50\% hammers). Unfortunately, the earliest convenient power source is the coal
plant. A factory plus coal plant will inflict severe unhealthiness on a city. Mitigating this effect with environmentalism
and/or hospitals and supermarkets should be a top priority after the factories are down.

\end{document}
